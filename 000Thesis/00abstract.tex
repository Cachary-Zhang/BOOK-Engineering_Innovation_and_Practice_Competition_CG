\begin{abstract}
    工程创新与实践能力是现阶段大学生体现学科知识综合应用与协同平衡的能力,可帮助广大同学加深对机械、电子、控制等方面课程的理解。在实际生活,该项能力不管是求职,抑或升学都有着举足轻重的作用。

    工程创新与实践大赛作为训练该项能力的一场重要竞赛,要求各位参赛选手坚持理论与实践相结合、学科专业相交叉、校企协同共创新、理工人文互惠通,发展成为卓越工程技术后备人才。

    本比赛可用于综合实践锻炼、保研加分、技术类科研设计等场景。

    然而,对于绝大多数参赛选手,均为大一/大二/大三同学,对如何0基础着手准备这个比赛存在很大疑问,因此,本竞赛指导旨在“从0出发——手把手教会你如何开展工程创新与实践大赛,做到切实拥有工程创新实践能力”。
    
    本教程共分为三大板块,\textbf{新能源车赛道;智能+赛道;虚拟仿真/企业运营赛道。}为不同学科基础同学提供可靠的竞赛指导方案。

    本书由张子涵、张晓敏、李航、尹国鹏编写。由于作者水平有限,错误和不足之处在所难免,希望各位专家和读者批评指正。同时也希望各位读者以本指导方案为设计基础,进行发散创新,获得属于自己的那份竞赛宝典~

    最后,预祝各位参赛选手竞赛顺利!!!

\end{abstract}