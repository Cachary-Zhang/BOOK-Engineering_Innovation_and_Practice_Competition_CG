
\section{新能源车赛道}
\subsection{绪论}
% \subsubsection{11111}
新能源车赛道是由本比赛创始之初的\textbf{势能驱动车/热能驱动车}赛道升级调整得来,该项目现阶段比赛类型包括\textbf{太阳能小车}和\textbf{温差小车}两种。故名思义,新能源赛项是通过使用太阳能发电,和温差化石能源发电,为小车提供动力,从而使得小车能够完成规定半开放轨迹的运动。

如下图\ref{fig:1}所示,第七届工创赛依旧延续往届运动轨迹,要求参赛者设计的小车能够通过凸轮/棘轮/槽轮等方式,在完全耗尽势能/酒精之前,能够尽可能多得完成绕桩圈数。根据评分标准,“8”字绕桩方法比环形绕桩方法得分更高,但同时“8”字绕桩对凸轮设计的要求也更高。

反观第八届工创赛,赛提要求,小车从瑞金出发,自拟路径完成开环形式的小车运动轨迹,直至抵达延安。在这一过程中,小车需按照要求经过沿途打卡点\textbf{(允许漏点,但不允许重复经过同一个打卡点)},相比之下,第八届运动轨迹更加开放,设计方法更多元,能够给参赛选手更多的设计体验,同时也为凸轮设计带来了更大的难度(这里需要预备机械设计相关知识:在设计过程中凸轮压力角对凸轮轮廓有很大影响,这一部分我们在后续小节中会着重讲解)。
\begin{figure}[htbp]
    \centering
    \includegraphics[width = \linewidth]{../03Picture/赛事发展.png}
    \caption{无碳小车赛事变化}
    \label{fig:1}
\end{figure}

工程实践与创新能力大赛要求参赛选手从\textbf{车体结构设计,凸轮计算以及调车方法}三个方面进行着手研究,旨在综合培养选手的实践与创新能力。

本章节将逐层递进,从机械设计原理入手,通过理论与案例相结合的方法,手把手教会你如何开展小车结构设计;如何通过编程方式高效完成凸轮设计;如何在小车制作完成后进行调车,并解决赛场上的突发性事件。

同时,我也希望学有余力的同学,大胆创新,让自己从这个比赛中学到能多技能。

\vspace{30pt}
\subsection{机械结构设计}
\subsubsection{整体车架计算与设计}

\subsubsection{前叉设计}

\subsubsection{传动机构的计算与设计}

\subsubsection{缓启动装置设计}


\subsection{凸轮设计与计算}
\subsubsection{凸轮设计基础}
\subsubsection{凸轮计算方法}
\subsubsection{凸轮计算中常见的问题及解决方法}
\subsubsection{SoildWorks仿真验证}


\subsection{MATLAB程序设计}
\subsubsection{基础版程序设计}
\subsubsection{进阶版程序设计}





\subsection{电路设计}
\subsubsection{打卡/语音播报模块电路设计}
\subsubsection{超级电容模块电路设计}




\subsection{调车指南}
\subsubsection{案例1:xxx}
\subsubsection{案例2:xxx}
\subsubsection{案例3:xxx}
\subsubsection{案例4:xxx}



